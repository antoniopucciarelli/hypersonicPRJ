% https://github.com/martinhelso/UiB
\documentclass[UKenglish]{beamer}

\usetheme{UiB}

\usepackage[utf8]{inputenx} % For æ, ø, å
\usepackage{csquotes}       % Quotation marks
\usepackage{microtype}      % Improved typography
\usepackage{amssymb}        % Mathematical symbols
\usepackage{mathtools}      % Mathematical symbols
\usepackage[absolute, overlay]{textpos} % Arbitrary placement
\setlength{\TPHorizModule}{\paperwidth} % Textpos units
\setlength{\TPVertModule}{\paperheight} % Textpos units
\usepackage{tikz}
\usetikzlibrary{overlay-beamer-styles}  % Overlay effects for TikZ


\author{Antonio~Pucciarelli \and Raffaele~Tirotta}
\title{\textsc{Mars Landing}}
\subtitle{\textsc{Hypersonic Flows project}}

\begin{document}


\section{Overview}
% Use
%
%     \begin{frame}[allowframebreaks]{Title}
%
% if the TOC does not fit one frame.
\begin{frame}{Table of contents}
    \tableofcontents[currentsection]
\end{frame}


\section{Mathematics}
\subsection{Theorem}


\begin{frame}{Mathematics}
    \begin{theorem}[Fermat's little theorem]
        For a prime~\(p\) and \(a \in \mathbb{Z}\) it holds that \(a^p \equiv a \pmod{p}\).
    \end{theorem}

    \begin{proof}
        The invertible elements in a field form a group under multiplication.
        In particular, the elements
        \begin{equation*}
            1, 2, \ldots, p - 1 \in \mathbb{Z}_p
        \end{equation*}
        form a group under multiplication modulo~\(p\).
        This is a group of order \(p - 1\).
        For \(a \in \mathbb{Z}_p\) and \(a \neq 0\) we thus get \(a^{p-1} = 1 \in \mathbb{Z}_p\).
        The claim follows.
    \end{proof}
\end{frame}


\subsection{Example}


\begin{frame}{Mathematics}
    \begin{example}
        The function \(\phi \colon \mathbb{R} \to \mathbb{R}\) given by \(\phi(x) = 2x\) is continuous at the point \(x = \alpha\),
        because if \(\epsilon > 0\) and \(x \in \mathbb{R}\) is such that \(\lvert x - \alpha \rvert < \delta = \frac{\epsilon}{2}\),
        then
        \begin{equation*}
            \lvert \phi(x) - \phi(\alpha)\rvert = 2\lvert x - \alpha \rvert < 2\delta = \epsilon.
        \end{equation*}
    \end{example}
\end{frame}


\section{Highlighting}
\SectionPage


\begin{frame}{Highlighting}
    Sometimes it is useful to \alert{highlight} certain words in the text.

    \begin{alertblock}{Important message}
        If a lot of text should be \alert{highlighted}, it is a good idea to put it in a box.
    \end{alertblock}

    It is easy to match the \structure{colour theme}.
\end{frame}


\section{Lists}


\begin{frame}{Lists}
    \begin{itemize}
        \item
        Bullet lists are marked with a red box.
    \end{itemize}

    \begin{enumerate}
        \item
        \label{enum:item}
        Numbered lists are marked with a white number inside a red box.
    \end{enumerate}

    \begin{description}
        \item[Description] highlights important words with red text.
    \end{description}

    Items in numbered lists like \enumref{enum:item} can be referenced with a red box.

    \begin{example}
        \begin{itemize}
            \item
            Lists change colour after the environment.
        \end{itemize}
    \end{example}
\end{frame}


\section{Effects}


\begin{frame}{Effects}
    \begin{columns}[onlytextwidth]
        \begin{column}{0.49\textwidth}
            \begin{enumerate}[<+-|alert@+>]
                \item
                Effects that control

                \item
                when text is displayed

                \item
                are specified with <> and a list of slides.
            \end{enumerate}

            \begin{theorem}<2>
                This theorem is only visible on slide number 2.
            \end{theorem}
        \end{column}
        \begin{column}{0.49\textwidth}
            Use \textbf<2->{textblock} for arbitrary placement of objects.

            \pause
            \medskip

            It creates a box
            with the specified width (here in a percentage of the slide's width)
            and upper left corner at the specified coordinate (x, y)
		(here x is a percentage of width and y a percentage of height). \cite{Artin1966}
        \end{column}
    \end{columns}
    
    \begin{textblock}{0.3}(0.45, 0.55)
        \includegraphics<1, 3>[width = \textwidth]{UiB-images/UiB-emblem}
    \end{textblock}
\end{frame}


\section{References}


\begin{frame}[allowframebreaks]{References}
    \begin{thebibliography}{}

        % Article is the default.
        \setbeamertemplate{bibliography item}[book]

        \bibitem{Hartshorne1977}
        Hartshorne, R.
        \newblock \emph{Algebraic Geometry}.
        \newblock Springer-Verlag, 1977.

        \setbeamertemplate{bibliography item}[article]

        \bibitem{Helso2020}
        Helsø, M.
        \newblock \enquote{Rational quartic symmetroids}.
        \newblock \emph{Adv. Geom.}, 20(1):71--89, 2020.

        \setbeamertemplate{bibliography item}[online]

        \bibitem{HR2018}
        Helsø, M.\ and Ranestad, K.
        \newblock \emph{Rational quartic spectrahedra}, 2018.
        \newblock \url{https://arxiv.org/abs/1810.11235}

        \setbeamertemplate{bibliography item}[triangle]

        \bibitem{AM1969}
        Atiyah, M.\ and Macdonald, I.
        \newblock \emph{Introduction to commutative algebra}.
        \newblock Addison-Wesley Publishing Co., Reading, Mass.-London-Don
        Mills, Ont., 1969

        \setbeamertemplate{bibliography item}[text]

        \bibitem{Artin1966}
        Artin, M.
        \newblock \enquote{On isolated rational singularities of surfaces}.
        \newblock \emph{Amer. J. Math.}, 80(1):129--136, 1966.

    \end{thebibliography}
\end{frame}


\end{document}
